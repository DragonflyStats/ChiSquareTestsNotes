
\documentclass[a4]{beamer}
\usepackage{amssymb}
\usepackage{graphicx}
\usepackage{subfigure}
\usepackage{newlfont}
\usepackage{amsmath,amsthm,amsfonts}
\usepackage{beamerthemesplit}
\usepackage{pgf,pgfarrows,pgfnodes,pgfautomata,pgfheaps,pgfshade}
\usepackage{mathptmx}  % Font Family
\usepackage{helvet}   % Font Family
\usepackage{color}

\mode<presentation> {
 \usetheme{Default} % was
 \useinnertheme{rounded}
 \useoutertheme{infolines}
 \usefonttheme{serif}
 %\usecolortheme{wolverine}
% \usecolortheme{rose}
\usefonttheme{structurebold}
}

\setbeamercovered{dynamic}

\title[Stats-Lab.com]{\LARGE Introduction to Statistics and Probability \\ {\Large Chi-Square Test : }}
\author[Kevin O'Brien]{Kevin O'Brien}
\date{Spring 2014}


\renewcommand{\arraystretch}{1.5}

\begin{document}


\begin{frame}
\titlepage
\end{frame}

%---------------------------------------------%
\begin{frame}
	\frametitle{Chi-Square Test of Association}
	% Writing
	
	\Large
	\textbf{Cell$_{(1,1)}$}
	\begin{center}
		\begin{tabular}{|c|c|c|c|c|}
			\hline & Cat X & Cat Y & Cat Z & Total  \\ \hline
			Cat A & \alert{\ldots}& 60 &  & 200\\ \hline
			Cat B & \phantom{space}& \phantom{space} & \phantom{space} & 400 \\ \hline
			Total & 150 & 180 & 270 &  \textbf{600}\\ \hline
			
		\end{tabular} 
	\end{center}
	
\end{frame}
%---------------------------------------------%

\begin{frame}
	\frametitle{Chi-Square Test of Association}
	\Large
	\textbf{Cell$_{(1,1)}$}
	\begin{itemize}
		\item Row 1 : Row Total = 200
		\item Column 1 : Column Total = 150
		\item Overall Total = 600
	\end{itemize}
	\bigskip
	
	
	Expected value for Cell$_{(1,1)}$
	
	\[ E_{(1,1)} = \frac{200 \times 150}{600} = \frac{30,000}{600} = 50 \]
\end{frame}
%---------------------------------------------%
\begin{frame}
	\frametitle{Chi-Square Test of Association}
	% Writing
	
	\Large
	Expected values for all of the other cells can be computed the same way.
	\begin{center}
		\begin{tabular}{|c|c|c|c|c|}
			\hline & Cat X & Cat Y & Cat Z & Total  \\ \hline
			Cat A & 50 & 60 &  & 200\\ \hline
			Cat B & \phantom{space}& \phantom{space} & \phantom{space} & 400 \\ \hline
			Total & 150 & 180 & 270 &  \textbf{600}\\ \hline
			
		\end{tabular} 
	\end{center}
	
	
\end{frame}

\begin{frame}
\frametitle{Chi-Square Test of Association}

\Large
\textbf{Critical Value}

\begin{itemize}
\item Signifiance Level $\alpha$
\item Degrees of freedom $\nu$ (also referred to as $d.f.$)
\end{itemize}

\end{frame}
%---------------------------------------------%
\begin{frame}
\frametitle{Chi-Square Test of Association}
\Large
\vspace{-1cm}
The degrees of freedom $\nu$

\[ \nu  = (r-1) \times (c-1) \]


\begin{itemize}
\item $r$ number of rows in frequency table
\item $c$ number of columns in frequency table
\end{itemize}

\end{frame}

%---------------------------------------------%
\section{Goodness of Fit}
\begin{frame}
\frametitle{Chi-Square Goodness of Fit}
\Large
\vspace{-1cm}
30 Observations are thought to be indepedent realizations of a Poisson Random Variable $X$ with mean 6.

\begin{itemize}
\item $X \leq 4$ 10 occurrences
\item $4 < X \leq 7$ 8 occurrences
\item $ X >7$ 12
\end{itemize}
Test the hypothesis that the observations on $X$ are from a Poisson Distribution with mean 6.
\end{frame}
%---------------------------------------------%
\end{document}
